\documentclass[11pt,journal,compsoc]{IEEEtran}
\linespread{2}
\ifCLASSOPTIONcompsoc
  % IEEE Computer Society needs nocompress option
  % requires cite.sty v4.0 or later (November 2003)
  \usepackage[nocompress]{cite}
\else
  % normal IEEE
  \usepackage{cite}
\fi

% IEEEtran.cls V1.7 and later.






% *** PDF, URL AND HYPERLINK PACKAGES ***
%
%\usepackage{url}
% url.sty was written by Donald Arseneau. It provides better support for
% handling and breaking URLs. url.sty is already installed on most LaTeX
% systems. The latest version and documentation can be obtained at:
% http://www.ctan.org/pkg/url
% Basically, \url{my_url_here}.

% correct bad hyphenation here
\hyphenation{op-tical net-works semi-conduc-tor}


\begin{document}
%
\title{Internet Piracy in the \\ Age of SOPA and PIPA}
%
\author{Daniel~Cherney,~\IEEEmembership{Member,~ACM,}
        Melissa~Riddle,~\IEEEmembership{Member,~ACM}%
    
\IEEEcompsocitemizethanks{\IEEEcompsocthanksitem D. Cherney is a Cybersecurity student at
	California State University.\protect\\
% note need leading \protect in front of \\ to get a newline within \thanks as
% \\ is fragile and will error, could use \hfil\break instead.
E-mail: dcherney@acm.org
\IEEEcompsocthanksitem M. Riddle is a Math and Computer Science student at
	California State University.\protect\\
	% note need leading \protect in front of \\ to get a newline within \thanks as
	% \\ is fragile and will error, could use \hfil\break instead.
	E-mail: melissariddle@csu.fullerton.edu}% <-this % stops an unwanted space
\thanks{Manuscript received September 16, 2017; revised .}}


% The paper headers
\markboth{Technical Writing for Computer Science, Fall~2017}%
{Shell \MakeLowercase{\textit{et al.}}: Internet Piracy in the Age of SOPA and PIPA}

\IEEEtitleabstractindextext{%
\begin{abstract}
SOPA and PIPA are the next steps towards ending net neutrality. However, not many people have a good understanding of the consequences of these bills. Translating these documents from legal terms to those anybody could use needs to be done so that the public can have complete understanding. Starting with the laws themselves, we have worked outward to dissect SOPA and PIPA. We expect that with the knowledge of these bills and their possible consequences, the informed public will focus on what their government is trying to put into law.
\end{abstract}

\begin{IEEEkeywords}
SOPA, PIPA, net neutrality, DNS, intellectual property, IP addressing
\end{IEEEkeywords}}

\maketitle

\IEEEdisplaynontitleabstractindextext

\IEEEpeerreviewmaketitle



\IEEEraisesectionheading{\section{Introduction}\label{sec:introduction}}

\IEEEPARstart{T}{he} Stop Online Piracy Act (SOPA) and Protect IP Act (PIPA) are bills in the U.S. House of Representatives and U.S. Senate respectively that seek to end piracy on the internet. The implications of these legislation pieces are far reaching. Their effects would be felt around the globe. The United States is attempting to exert their power on a domain that does not belong to any jurisdiction. The drawbacks of trying to regulate the internet far outweigh the possibility of stopping piracy online. The world is more connected now than ever before. According to internet usage statistics, more than half of the world's population has access to the internet. It is essential that all of these people have equal access to the internet. SOPA and PIPA would allow the U.S. government and internet service providers (ISPs) to disrupt access to specific websites. On the other hand, internet piracy is at an all-time high. Companies that produce music and movies are seeing lower profits because of internet users distributing the intellectual property. The congressmen and congresswomen who wrote SOPA and PIPA have good intentions. These laws are not the proper solution to the problem.


\subsection{What is the purpose of the SOPA and PIPA?}
Subsection text here.

\subsection{How does SOPA attempt to stop piracy?}
SOPA attempts to stop online piracy through means which threaten a free and open internet. The first way that these bills try to protect United States intellectual property laws is by enforcing DNS filtering. DNS filtering would prevent US IP addresses from accessing foreign websites accused of helping contribute to piracy. The potential for abuse by copyright holders is very high. Measuring United States intellectual property laws against the laws of sovereign nations elucidates a clear intent to exert further the United States' will internationally. SOPA also authorizes \textit{in rem} lawsuits in U.S. courts against domains that are found to be in violation. These bills state that any person which U.S. courts wouldn't naturally have jurisdiction over (i.e., anyone who is a citizen of any country besides the United States), the judge would gain the power to indict them. Search engines would be required to remove the website from its database of sites that return search results. It would also disallow internet advertising networks from advertising on the specific website, and that website could not publish advertisements to its domain on other sites. These laws would work to cut off funding for websites so it would be impractical to keep the website running. There would also be a prohibition of Visa or Mastercard transactions on the web page in question. If any group or organization refused to cooperate with the U.S. government, they would be subject to what is known as an enforcement proceeding. SOPA attempts to stop piracy by filtering out the websites accused of facilitating piracy. By forcing groups or organizations to cooperate with the U.S. government, the lawmakers are trying to make the internet bend to their will. The goal of the legislation is to shape the internet so that any illegal material is not accessible to the public. SOPA attempts to stop piracy by filtering out the websites accused of facilitating piracy.

\subsection{What was congress trying to accomplish?}
Subsection text here.

\subsection{Why would there be support for such bills?}
The primary support for the bill is currently media companies. These companies believe that piracy negatively affects revenue. These corporations petition to get these bills passed in an attempt to increase their income. Places like Hollywood spend a lot of money on the production of several blockbuster movies. Piracy, to them, means one less DVD sold and one less ticket sold. There would also be a lot of interest in this law by those who wish to expand the jurisdiction of the United States. SOPA and PIPA would allow the United States to develop its powers in prosecuting those who violate the United States' copyright laws. However, it is unsure whether or not other nations would be cooperative with the United States in this endeavor. However, there was a significant backlash about these bills. Several large websites like Wikipedia and Google had what is known as a "blackout" in protest of these laws. For many users of the Internet, it is a public enterprise that can be used regardless of your country of origin. The Internet community believes that the internet is too big for regulation by only one state. They think that these laws create severe limitations on the internet. The United States government condemned these blackouts. Most of the support for these bills come in the form of musicians or actors representing Hollywood. These people wish to see their revenue increase, so they want to disrupt the international organization known as the internet. The internet is not the problem, though. The question that these companies are trying to address is how to prosecute people who distribute their intellectual property to others for free. Many people would say that they do not approve of stealing. It is a characteristic cultural value in the United States. A layperson might agree that online piracy should stop, but they will also have a great misunderstanding of the internet. Internet culture is a worldwide entity. Corporations that wish to protect their intellectual property and misinformed citizens are the main supporters of these two acts.

\subsection{What is the status of SOPA and PIPA right now?}
Subsection text here.

\subsection{Replacement Policy}
Creating a replacement policy for either SOPA or PIPA that both sides agree on would be extremely difficult. People who did not want these bills passed were fighting for freedom of the internet from regulation. These same people might agree with U.S. Intellectual Property laws, but they do not believe the internet is a place for national policy-making. A replacement law suggests that the U.S. Government is still trying to regulate the internet. 
\indent Online piracy is a crime of intellectual property. Intellectual property law is a lot harder to enforce internationally. Every country has its laws regarding how intellectual property should be treated. The differences in these requirements lead to an unclear situation regarding the implementation of legislation. If a man in China is violating U.S. intellectual property law, he would not be subject to prosecution. The only case in which he would be subject to prosecution would be if China extradited the man to the United States. Intellectual property is relative to jurisdiction. Cultural differences lead to the sentiment that a nation should not be able to regulate the internet. The internet is international and outside of the authority of any one country. 
\indent An approach to solving online piracy would not be as active as the current proposals. Advocates for net neutrality would prefer that the U.S. government only prosecute U.S. citizens. An argument point for this would be that the United States would not be stepping outside of its jurisdiction. It would be easier for U.S. officials to trace the IP address of offenders and arrest pirates. Investigators could monitor internet activity of individual computers to build a case against people who violate U.S. intellectual property law. Corporations could also help lower the piracy rate by making their music or movies available on streaming services. Cooperation and transparency would be a must between the public, corporations and government.
\indent The difference in these viewpoints leads to a conflict. One side of the conversation does not want regulation of the internet. The other party is pushing for control of the internet. Lawmakers could try to introduce a less strict law to test the public opinion in a couple of years. In its current state, it would be near impossible for lawmakers to pass regulation on the internet in the way SOPA and PIPA do. 


\section{Conclusion}
Online piracy is an issue for corporations and the internet. The current attempts to try to stop piracy are not sufficient enough to ensure that the web remains free. SOPA and PIPA are examples of legislation that fight against an international internet. Even though these bills are dead in the house, the precedent that these measures establish is scary. Piracy is a problem the internet has always encountered. It is an unfortunate side effect of having an international forum. The current benefits the internet provides far outweighs any consideration of regulation.  If there are attempts to regulate the internet in the future, there will be significant changes in the approach to the rule. The internet community proved that it is ready and willing to defend the freedoms which it enjoys. Corporations, such as Google, are prepared to protest against these policies. Internet users in America will continue to appreciate the freedoms they have online. Internationally, the fight will continue for a free and open internet. The death of SOPA and PIPA is a step in the correct direction.

\appendices
\section{}
Appendix one text goes here.


% use section* for acknowledgment
\ifCLASSOPTIONcompsoc
  % The Computer Society usually uses the plural form
  \section*{Acknowledgments}
\else
  % regular IEEE prefers the singular form
  \section*{Acknowledgment}
\fi


The authors would like to thank Professor Pouya Radfar, Richard Stallman and the Free Software Foundation.


% Can use something like this to put references on a page
% by themselves when using endfloat and the captionsoff option.
\ifCLASSOPTIONcaptionsoff
  \newpage
\fi


% trigger a \newpage just before the given reference
% number - used to balance the columns on the last page
% adjust value as needed - may need to be readjusted if
% the document is modified later
%\IEEEtriggeratref{8}
% The "triggered" command can be changed if desired:
%\IEEEtriggercmd{\enlargethispage{-5in}}

% references section

% can use a bibliography generated by BibTeX as a .bbl file
% BibTeX documentation can be easily obtained at:
% http://mirror.ctan.org/biblio/bibtex/contrib/doc/
% The IEEEtran BibTeX style support page is at:
% http://www.michaelshell.org/tex/ieeetran/bibtex/
%\bibliographystyle{IEEEtran}
% argument is your BibTeX string definitions and bibliography database(s)
%\bibliography{IEEEabrv,../bib/paper}
%
% <OR> manually copy in the resultant .bbl file
% set second argument of \begin to the number of references
% (used to reserve space for the reference number labels box)

\begin{thebibliography}{1}

\bibitem{IEEEhowto: MMG} 
Internetworldstats.com. (2017). World Internet Users Statistics and 2017 World Population Stats. [online] Available at: 				http://www.internetworldstats.com/stats.htm [Accessed 1 Oct. 2017].
\bibitem{IEEEhowto: SOPA}
Stop Online Piracy Act (H.R. 3261), vol. 1-205. Washington DC: Library of Congress, 2011.
\bibitem{IEEEhowto: PIPA}
Protect IP Act (S.968), vol. 1-8. Washington DC: Library of Congress, 2011.
\bibitem{IEEEhowto: Article}
C. Hill, "Digital piracy: Causes, consequences, and strategic responses", Asia Pacific Journal of Management, vol. 24, no. 1, pp. 1-18, 2007.
\bibitem{IEEEhowto: netneutrality}
K. Frenkel, "Internet Companies Halt Anti-Privacy Bills", Association for Computing Machinery. Communications of the ACM, vol. 55, no. 3, 2012.

\end{thebibliography}

% biography section
% 
% If you have an EPS/PDF photo (graphicx package needed) extra braces are
% needed around the contents of the optional argument to biography to prevent
% the LaTeX parser from getting confused when it sees the complicated
% \includegraphics command within an optional argument. (You could create
% your own custom macro containing the \includegraphics command to make things
% simpler here.)
%\begin{IEEEbiography}[{\includegraphics[width=1in,height=1.25in,clip,keepaspectratio]{mshell}}]{Michael Shell}
% or if you just want to reserve a space for a photo:

\begin{IEEEbiographynophoto}{Daniel Cherney}
Daniel is currently a Computer Science student at California State University at Fullerton. He is studying Cybersecurity for a world that needs more people to keep information safe. Daniel is proud to be a supporter and contributor to several open source software projects. 
\end{IEEEbiographynophoto}

% if you will not have a photo at all:
\begin{IEEEbiographynophoto}{Melissa Riddle}
Biography text here.
\end{IEEEbiographynophoto}


\end{document}


